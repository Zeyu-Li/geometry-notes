\documentclass[11pt]{article}
\usepackage[utf8]{inputenc}
\usepackage{amsmath}
\usepackage{amssymb}
\usepackage{graphicx}

\title{Geometry Notes}
\author{Andrew }
\date{September 2021}

\begin{document}

% \maketitle

\section{Basics}
\subsection{Logic \& Proofs}

\begin{itemize}
  \item If then: \textbf{antecedent} $\to$ \textbf{consequent} [$P\to Q$]
  \item Converse: consequent $\to$antecedent [$Q\to P$]
  \item Contrapositive  [$\sim Q\to \sim P$]
  \item Negation/denial [$\sim (P\to Q) \iff P\land \sim Q$]
  \item If and only if statements [$P\iff Q$] is logically equivalent to [$P\to Q$ and $Q\to P$]
\end{itemize}

\subsubsection{Proof Strategies }

\begin{itemize}
  \item Directly (usually involves a few cases)
  \item Contrapositive
  \item Contradiction: proof there is a contradiction if you assume the negation
  \item Counterexample (only 1 is needed)
  \item Bonus: induction $\to$ prove for base case, assume for n, prove for n+1
\end{itemize}

\subsection{Basics}
\subsubsection{Notation}

\begin{itemize}
  \item Ray $\overrightarrow{AB}$ goes from A past B to infinity
  \item Line $\overleftrightarrow{AB}$ goes past A and B to infinity
  \item $AB$ is just a line going from A to B
  \item Lines that have no intersection point are parallel and distinct [$P|| Q$]
  \item Lines that have a intersection point are non-parallel [$P\nparallel Q$]
  \item More than 1 intersection point is considered the same line
  \item Consider line S to be an infinite set of points, if point X in line S, then it is $X\in S$
  \item Intersection of lines will be $S\cap T$ 
  \item If lines do not intersect then it is said that $S\cap T=\emptyset$ (ie the intersection of the two sets/lines is the trivial/null set)
\end{itemize}


\subsubsection{Angles}

\begin{itemize}
  \item A full rotation is $360^{\circ}$
  \item A line is $180^{\circ}$
  \item Right triangle is $90^{\circ}$
  \item $\angle BAC$ is the interior of the rays $\overrightarrow{AB}$ and $\overrightarrow{AC}$. Shorthand is $\angle A$
  \item Outside angle is the reflex angle
  \item When two lines cross, there are 2 pairs of opposite angles (this is known at the supplementary angle)
\end{itemize}

The following are two lines l and m with a \textbf{transversal} t

\noindent\includegraphics{pic1.png}
\noindent\includegraphics{pic2.png}

\subsubsection{Shapes}
\begin{itemize}
  \item Circle at point A with radius r is $C(A,r)$
  \item $\triangle ABC$ have vertices A, B, C 
  \item Names for the sides of a triangle are lower case and are the opposite to the angle it corresponds with
  \item The sum of the lengths of any two sides of a triangle is always greater than the length of the remaining side
  \item The sum of the lengths of any two sides of a triangle is always greater than the length of the remaining side
\end{itemize}

\section{Intro}
\subsubsection{Axioms}
\begin{itemize}
  \item \textbf{The Parallel Postulate}: for alternate angles image above, if angles 1 and 2 sum to less than 180 then l,m must intersect to the right of t 
  \item \textbf{Playfair's Axiom}: Given line l and point P, there exists only 1 line though P parallel to l
\end{itemize}
\subsubsection{Big Boy Theorems}
\textbf{The Transversal Angles Theorem}: if any of the following is true then they are are true about the transversal t that passes two distinct lines l and m
\begin{enumerate}
    \item The adjacent angles sum to $180^{\circ}$
    \item The alternate angles are equal
    \item The alternate exterior angles are equal
    \item The corresponding angles are equal
    \item The lines l and m are parallel
\end{enumerate}

\noindent \textbf{The Triangle 180 Theorem}: The interior angles of a triangle sum to 180

\includegraphics{pic3.png}
\begin{itemize}
  \item Blue = interior angles
  \item Green = opposite to its interior
  \item Red = exterior angles
\end{itemize}
\noindent exterior angle of a triangle is equal to the sum of the two interior angles on the opposite side

\section{Congruent }
Congruent occurs when two objects are the same but just different position in space, possibly rotated or flipped
\begin{itemize}
  \item $\triangle ABC \equiv \triangle DEF$ iff all of the following hold
  \item $\angle A \equiv \angle D$ and $\angle B \equiv \angle E$ and $\angle C \equiv \angle F$ 
  \item $BC \equiv EF$ and $AC \equiv DF$ and $AB \equiv DE$ 
\end{itemize}
\subsubsection{Triangle Congruences}
\begin{itemize}
  \item SAS
  \item ASA
  \item AAS
  \item SSS
  \item ASS* (For second side (S) to be longer than the first)
  \item RSH
\end{itemize}
If an triangle is \textbf{isosceles} if two side are equal length (ie 2 congruent sides)

\noindent \textbf{The Angle Side Inequality Theorem}: For triange ABC, $\angle B > \angle C \iff AC > AB$

\subsubsection{Right Triangles}
The oppose of the 90 degree change of the right triangle is the \textbf{hypotenuse}

\noindent\includegraphics{pic4.png}
\noindent Trivial by inspection (Above)


\noindent\includegraphics{pic5.png}
\noindent Red is \textbf{altitude}, green is \textbf{median}, and purple is \textbf{angle bisector}

\noindent\textbf{The Isosceles Implies Theorem}: For isosceles triangle ABC, the median from A, altitude, and the bisector of angle A are all the same

\noindent\textbf{The implies Isosceles Theorem}: Reverse of the previous but only two of the three are required to imply the last

\subsection{Quadrilaterals}
\begin{itemize}
  \item Order matters when naming quads
  \item 3 Types, convex and non-convex that are regular and non-simple
  \item Convex have 2 interior diagonals, non-convex has 1 interior 1 exterior and non-simple have 2 exterior
  \item Parallelogram is a quadrilateral whose opposite sides are parallel
  \item Simple quadrilaterals have interior angles than sum up to 360
  \item Non-simple quadrilaterals sum to $< 360$
\end{itemize}

\noindent \textbf{Parallelogram Implies Theorems}: For parallelogram ABCD
\begin{enumerate}
    \item Opposite sides and angles are congruent
    \item The diagonals bisect each other 
\end{enumerate}

\noindent \textbf{Implies Parallelogram Theorems}: A simple parallelogram ABCD if any of the following conditions are meet
\begin{enumerate}
    \item Opposite sides or angles are congruent
    \item The diagonals bisect each other 
    \item One pair of opposite sides are parallel and congruent
\end{enumerate}

\noindent \textbf{The Midline Theorem}: If points P and Q are midpoints of AB and AC respectably, then for $\triangle ABC$, $PQ||BC$ and is half the length of BC [$PQ\overset{\shortparallel}{=}\frac{BC}{2}$]

\noindent \textbf{The Midpoint Theorem}: Similar to Midline theorem but instead of point Q, we take line l that is parallel to BC and intercepts P. This implies that the new point Q is midpoint of AC

\end{document}
